\label{sec:impltools}
These differ from design tools in that these are tools not thought of during design but which were necessitated by something in the process of implementation.

\subsection{Dockerpy}
\label{impl:dockerpy}
Dockerpy \citep{dockerpy} is a python wrapper for docker functionality which was used in \emph{rmt} to interact with docker in the client application.

\subsection{MySQL}
MySQL and the accompanying MySQLdb python module were used to create and manage the database for \emph{rmt}.
This was used as a matter of personal preference over alternatives such as PostgreSQL and because \verb%web.py%'s database module supports it.

\subsection{Alternative Text Editors}
During implementation it was clear that PyCharm was interfering with workflow in that its auto-indentation in HTML files was ignoring \verb%web.py%'s templating language.
Since python and the aformentioned templating language are so dependent on indentation for scope, it was faster to develop these in an alternative text editor.
Additionally when having to manipulate colours it was more beneficial to use a colour chooser than simply guess the hash values, so geany \citep{geany} was used to write some of the HTML files since it has a colour chooser built in and does not auto-indent.
Geany also makes it clear whether the indents are tabs or spaces, meaning there was little chance of interfering with the indents set by PyCharm.

\subsection{Bootstrap}

Bootstrap is a responsive front-end web framework designed to be used for web applications of all sizes and, thanks to the repsonsive element, can be viewed neatly on mobile devices.
While it was not a requirement to view \emph{rmt} on mobile devices, these were catered for as much as possible in the design of \emph{rmt} so that the general public could view the application if they were on University grounds and connected to the public University wireless network.

Bootstrap is also useful for reducing time spent working with CSS (Cascading Style Sheets) which determine the look and feel of a website as this can be tedious.
With this framework however most of the work is already done and any additional styling can be added for requirements.

For example in \emph{rmt} the list-group-item elements were altered to make the colours of states different to the panel headers and the additional class, \verb%list-group-item-dead% was added to represent a dead host, the source for which is as follows:

\begin{lstlisting}[caption={style.css: list-group-item additions}]
.list-group-item-dead {
  color: #a1a1a1;
  background-color: #A9A9A9;
}
a.list-group-item-dead {
  color: #D9D9D9;
}
a.list-group-item-dead .list-group-item-heading {
  color: inherit;
}
a.list-group-item-dead:hover,
a.list-group-item-dead:focus {
  color: #D9D9D9;
  background-color: #919191;
}
a.list-group-item-dead.active,
a.list-group-item-dead.active:hover,
a.list-group-item-dead.active:focus {
  color: #fff;
  background-color: #919191;
  border-color: #a1a1a1;
}
\end{lstlisting}

\subsection{web.py}
\verb%web.py% \citep{webpy} is a lightweight python web development framework meaning it has a distinct advantage over similar frameworks such as Django \citep{django} which were deemed too heavyweight to be useful to \emph{rmt}.
The requirement to be as light as possible motivated this decision more than anything else.
However while first looking for a framework this appeared to be well documented and appeared simple to use, very much unlike Django which follows very strict guidelines.

It also allowed for rapid prototyping with very little code so file sizes could be kept to a minimum and designs could be deployed quickly and easily.