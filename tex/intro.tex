\emph{rmt} or Ross's Monitoring Tool is a network monitoring tool 
designed to work with the University of Glasgow's Raspberry Pi 
Cloud. It's primary purpose is to provide an at-a-glance view of the 
status and resource usage of each Raspberry Pi and their respective 
Linux containers.

\section{University of Glasgow Raspberry Pi Cloud}
\label{intro:picloud}

The University of Glasgow's Raspberry Pi Cloud \citep{glapicloud, picloudblog} (which will now be 
referred to as PiCloud, not to be confused with PiCloud Inc. \citeyearpar{picloudinc}) is an effort by the University of Glasgow to replicate Warehouse Scale Computing (WSC) \citep{barroso2009datacenter} on a number of Raspberry Pi \citep{rasppi} devices to create a small-scale version of WSC.
This was implemented using a software stack consisting of various tools such as LXC \citeyearpar{lxc} running on the Pi devices, each communicating to a ``Pi Master'' which is a regular PC server.
Prior to the creation of \emph{rmt} no monitoring tool was in use, meaning that no user feedback was received when, for example, a host stopped communicating to the Pi Master.
Currently the PiCloud consists of 56 Raspberry Pi devices each running Raspbian Linux \citeyearpar{raspbian} with a custom kernel and docker \citeyearpar{docker} as part of the software stack to control the host's Linux containers (which will be referred to as simply ``containers'' from now on).
However the project initially intended to have a thousand (1000) hosts.
The devices themselves are organised into four towers of Lego bricks, each a separate colour: red, blue, yellow and grey; as can be seen in figure \ref{fig:pitowers}.

\begin{figure}[t]
	\centering
	\includegraphics[width=0.5\textwidth]{picloud_towers}
	\caption{Photograph of the lego-built Raspberry Pi towers}
	\label{fig:pitowers}
\end{figure}

\section{Raspberry Pi Board Specifications}
Raspberry Pi devices are credit card sized, so its resources are limited, the relevant specifications of the Raspberry Pi boards used in the PiCloud project are listed below:

\begin{center}
\begin{tabular}{c | c}
	CPU & RAM \\
	\hline \\
	700MHz & 256MB
\end{tabular}
\end{center}

Disk space is not an issue as each device has a 16GB memory card for disk space with only minimal terminal-based applications installed and no Graphical User Interface (GUI), which tend to use a relatively large amount of disk space.