From the beginning of the project there was a need for rapid prototyping since many of the design decisions would be made during implementation.
To enable this, the design methodology chosen was that of ``Scrum''.

\subsection{Scrum Overview}
Scrum is an agile software development methodology designed around the idea of work ``sprints''.
These consist of a unit of work, the design of which generally remains up to the ``Scrum Master'', who also oversees the proper use of the scrum methodology.
Each sprint is designed to add some feature to the implementation, whether it be an enhancement of previous features, or addition of entirely new features.
The length of time each sprint takes is variable, with some companies taking on daily scrum meetings to discuss sprints and others weekly.
This intuitively changes the amount of work expected to be done with each sprint.
There are three roles associated with scrums, one being the aforementioned ``Scrum Master'', the others being the ``Product Owner'' and the ``Development Team''.
The product owner represents the stakeholders, communicating their needs to the development team.
In this case the product owner was the supervisor of the project, David White, representing the other contributors to the PiCloud project, and also aided in managing sprints, thereby taking on some of the role of scrum master.
Intuitively, the development team was myself.

\subsection{Application of Scrum}
Each week, meetings were held between the development team and the product owner.
These meetings usually detailed the intended outcome of the next week's worth of work, otherwise known as a sprint.
For example the first sprint involved the following outcomes:

\begin{itemize}
	\item Get Raspian Linux to run on a development Pi device given to the development team
	\item If no problems occurred, try to get docker installed and start a container
	\item Read ``The datacenter as a computer: An introduction to the design of warehouse-scale machines''
\end{itemize}
